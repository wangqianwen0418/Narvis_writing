\section{Design Considerations}\label{sec:design_task}
In this section, we first describe our understanding of two groups of end users, i.e., editors and the general audiences. Then, we distil design tasks to guide our design and development of Narvis.

\subsection{User Perspectives and Methods}

Narvis aims to offer an efficient, expressive and friendly authoring tool for experts in data visualization, assisting them to create a slideshow to introduce advanced visual design to the general audience.  
Hence, we identify two different user perspectives: the editors and the general audience perspectives. Editors are visualization experts who have the need to create a slideshow to present visual designs. The general audiences have no prerequisite for visualization. They gain an understanding of a visual design through the slideshow created by the editor. 

To understand the current practice of making slideshows and the experience of reading tutorials, we collaborated with four teaching assistants (TAs) of a data visualization course and seven undergraduate students (UGs) taking this course. The four TAs are postgraduate students whose research interests are information visualization. Their duty of this course involves making slides to introduce visual designs from major publications in the field of visualization. The slides should cover fine-grained description to help students review them after class. The seven UGs have no prior experience in visualization, and have taken this course for no more than one month.  

We began by conducting semi-structured interviews with TAs, whom we identified as editors, and UGs, as general audiences. During the interviews with TAs, we asked their workflows of making slideshows and explaining visual design. To identify opportunities for Narvis, we also asked them to enumerate a list of challenges faced in the workflows. The interviews with UGs are semi-structured as well. We asked their comments in reading the slideshows and attending course lectures. Then, we used mind-mapping to find clusters in their comments that defined goals for an ideal slideshow. 


\subsection{Design Tasks}
Based on our observations and the interviews, we categorize six design tasks to guide the design of Narvis. Three tasks, denoted as DE, are originated from the interview with editors, i.e., the four TAs, and other three tasks (DA) are from the general audiences (UGs).

%Our premise is that the editor is familiar with a visualization design but unclear how to educate his audiences in an efficient way. Since the background of the audiences cannot be guaranteed, we assume they all have no prior knowledge in data visualization. 
\textbf{DE1. Emphasis on  efficiency.}
TAs used presentation tools, such as Power Point\footnote{https://office.live.com/start/PowerPoint.aspx}and KeyNote\footnote{http://www.apple.com/keynote/} to introduce visual designs. However, these tools are for general purpose and not tailored for visualization presentation. 
For example, "focus + context" techniques are widely used in data visualization to guide the user’s attention to the region of interest \cite{doleisch2003interactive, kosara2002focus+}. It requires exaggeration or suppression of the visual channels, like hue, luminance, sharpness, or size, of graphical elements, which are hard to perform in these common presentation tools. 
% Moreover, with the large number of visual encodings existing, she fails to determine an optimized order for explaining them. Even though she has many years of experience in data visualization, she never think about this issue before. 
%Splitting a visualization into fine-grained graphical elements and combining them into logic flow are tedious and time-consuming with existing tools. 
%Therefore, automatic decomposition of visualization and composition of graphical elements are necessary for facilitating design and production of a slideshow.
% TAs require an efficient approach to generate a slideshow introducing visual design.

\textbf{DE2. Suggest options.}
People with extensive experience in designing data visualization can have little knowledge about how to present a visual design. 
Thus, suggesting design options to them for creating a comprehensive slideshow is demanding. Many presentation tools already offer this kind of service. For example, Power Point Designer\footnote{https://support.office.com/en-us/article/About-PowerPoint-Designer-53c77d7b-dc40-45c2-b684-81415eac0617} automatically generate a list of professionally designed layouts based on the contents. However, these suggestions focus on general issues, especially on aesthetics, and give no special consideration for presenting a visual design.
For example, composing a clear narrative sequence from all the visual grammar employed can facilitate the perception process. A list of design options can help editors quickly ideate on how to organize the narrative sequence and convey the insights of a visual design. However, no available presentation tool supports such service. 
%\siwei{your logic is: other tools have no special consideration for doing sth. Reviewers may think: why they should have such consideration? You should emphasize these consideration is important in visualization because of some reason.}
%The complicated relationship, spatial and logical alike,  between different graphic elements is one of the biggest barriers that hinders the effective presentation of a visual design. Howe like Power Point Designer"
%By encouraging users to refine the clusters of graphic elements, to figure out relationship between visual units by editing a unit tree, and to modify the narrative templates we offered, we aim at motivating users to sort out the structure of a visualization, which will then result in a better-organized narrative sequence. 

\textbf{DE3. Collect feedback.} \textit{``When students read my slides, I do not know whether they can follow the logic, or whether the slides cover enough details for them to grasp the visual design.''}, one TA commented. 
% When giving a lecture, TAs are able to understand how their slides and logic are accepted through in-class interactions. 
Collecting feedbacks of audiences is crucial for editors to revise their slideshow, making it more understandable and attractive.  

%We implement a click collector which will automatically record click activity when audience view the explanation in our system. The click stream data can reveal information about viewing order, for example, the audience might skip one slide or go back to revisit another, the time spent on each slide, the degree of understanding at each part.The audience can also write down their comments while reviewing such an introduction slide show. Editor can adjust the narrative sequence as well the level of details based on these feedbacks. 

\textbf{DA4. Avoid information overload.} 
All UGs complained that they had experienced information overload in reading slides. 
%Though they might separate an informative slideshow into parts and read one part at a time, they might miss some when the information in one slide is overwhelming.
% We distinguish the amount of information in the slideshow and in one slide. Audience  However, 
When the information in one slide is overwhelming, it is common for they to miss important visual encodings. 
The slideshow should be well designed to ensure that the amount of new information in each slide is appropriate. 

\textbf{DA5. Avoid unconscious ignorance.}
Experts in data visualization, i.e. the TAs of the course, prone to treat some visual grammars as self-evident that need no  explanation. However, the lack of information confuses the UGs, who have no prior knowledge of visualization. 
Considering the importance of information integrity to a comprehensive slideshow, we need a mechanism to guarantee that all visual grammars are explained. 

\textbf{DA6. Keep the sense of overview}
%Grouping slides into sections, and inserting visual notice, like a progress bar, to indicate the overall structure is an effective and widely used presentation skill \siwei{if you say they are widely used, please cite}. 
%However, partly due to the lack of a clear narrative structure, this technique is rarely used in the presentation of a visualization, in spite of the demands from audiences.
%\siwei{do not say the technique is rarely used in visualization because it is not convincing. say how the technique can help visualization presentation, therefore it is important.}
When conveying a considerable amount of information to the audiences, they can easily get distracted or forget previous information. In this situation, taking some measures to inform the audiences of the overall structure and the current state can help the perception process. 
