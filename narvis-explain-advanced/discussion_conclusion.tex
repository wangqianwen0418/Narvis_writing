\section{Limitation and Discussion}
We are not pretending that Narvis is exclusive for all types of visualization design, considering the initial set of templates Narvis provides. However, by allowing users a high flexibility to create and edit templates, we believe its coverage  will quickly broaden as more and more users contribute their own templates to our library. 

To broaden the form of the input. Embedded in a visualization analysis tool. 

For evaluation: small sample size, only compared to powerpoint,
however, as with all qualitative studies with small sample size, these results should be treated with caution and DataClips warrants further evaluation to confirm if our initial insights apply more gener- ally. 

\section{Conclusion and Future Work}
In this paper, we present Narvis, an authoring tool for crafting introduction slideshows for the purpose of introducing new visual design. 
Inspired by previous work and our observation, we propose a constructive model for introducing data visualization. This model is realized as a library of templates in Narvis, which recommends narrative sequences for the introduction of different visual units and supports easy creation of different attention cues.To better guide the development of Narvis, we also interviewed target users to extract their requirements. 

We evaluate Narvis through xxxx,

and get the conclusion xxxxx,

In future work, we envision two main research directions. First, we will deepen our understanding of what makes compelling and comprehensive introduction slideshows for presenting visual design. We plan to ground our work on the data from Narvis, more specifically, the templates, slideshows uploaded by editor users and the click stream data from audience users. Second, we will improve the performance of Narvis by 1) adding url as a possible input source; 2)offering better support for organizing the structure of visual units; 3)

%% if specified like this the section will be committed in review mode
\acknowledgments{
The authors wish to thank A, B, C. This work was supported in part by
a grant from XYZ.}