\section{Limitation and Discussion}
Results of our study indicate that editors could make slideshow more efficiently with Narvis than with professional software. In addition, audience report that the perceived quality of slideshow generated with Narvis is higher than those created by editors from scratch. However, we identify four limitations existed in Narvis.

First, the library of templates cannot cover all innovative designs. In current prototype, we allow users to extend existing templates to alleviate this issue. 
% To deal with emerging designs, we plan to adopt more custom-defined templates by creating open platform that supports template sharing and downloading in future research. 
% Second, to decompose visualization precisely, users are required to import images with high resolution, which cannot be guaranteed. 
Second, the results of visualization decomposition highly depend on the quality of imported images. We plan to design more robust algorithm to tackle input with various resolution.
Third, interactive features are important and commonly-seen in modern visualization. However, Narvis is limited to static visual design. We plan to investigate narrative introduction of interactions in future research. The fourth limitation of our study is the evaluation. \siwei{I do not know what is our limitations in evaluation. Involve small number of editors? Or audience? The evaluation of quality of slideshow is limited? Our system is only compared to ppt? We try small number of designs? }

It worth noting that we propose a model for narrative presentation of data visualization. The model can be generalized to create other forms of presentation, such as narrative posters and data videos. \siwei{more generalization is better}


% We are not pretending that Narvis are exclusive for all types of visualization design. However, by allowing users a high flexibility to create and edit templates, we believe its coverage  will quickly broaden as more and more users contribute their own templates to our library. \par
% Metaphor for aesthetic purpose.
% Our algorithm, not applicable for 3d rendering picture. 
% In our model, we focus on statistic image and leave dynamic interaction at this time point, which is an important feature for advanced data visualization design. 


\section{Conclusion and Future Work}
We have presented Narvis, an authoring tool to generate slideshow for explaining visual designs. We developed Narvis based on our close collaboration with two editors and seven audience. \siwei{I am not sure about the logic workflow. Have you decided whether the tasks drive our model and system, or the tasks and model drives the system?}

In future work, we envision two main research directions. First, with the popularity of visualization toolkits, such as D3 \siwei{cite d3 paper}, an increasing number of visual designs are deployed online. To support a border usage scenario, we plan to equip Narvis with the ability to parse and analyze online visualizations. 
Second, we aim at iterative refinement of templates based on click behaviors of audience. Click behavior of audience is valuable for editors to improve and edit their slideshows.  
It will be great if Narvis can give suggestions to editors for revision of slideshow. 

%% if specified like this the section will be committed in review mode
