\section{Evaluation}
We conducted a user study to evaluate Narvis, and gain insights on how the authoring experience and output would compare to slideshows created with general presentation tools. Our study was a between-subjects design with two sample groups: one group of participants used Narvis, the other group used Powerpoint, commonly used to create presentation slideshow. We asked participants to produce an introduction slideshow that presents a visual design, based on a image of the visualization and the textual description we provided. 
We report our qualitative observations during this process, and provide insights on the quality of the slideshows generated from both groups by asking 20 volunteers to rate them.
\subsection{Participants}
We invites 4 experts in data visualization to this user study. Al of them have more-than-one-year experience in the design and implementation of data visualization. 
%\textbf{Audience }are novice in data visualization. They will review the slideshow produced by the experts, rate it, give subjective comments, and answer a series of questions to check their understanding of this visualization.\par
%For editors, we have 4 postgraduate students, aging between 22-30, and all of them have more than one year experience in data visualization.\par
%For audiences, we have 20 under graduate students, whose majors vary from business to biology. 
%According to the questionnaire, none of them have accessed advanced data visualization before. Only 13\% students know the tree map, and none can give a accurate explanation of theme river with topic splitting and merging.  \par
\subsection{Material}
We extract the visual design and the corresponding literature description from  a visualization design paper ``TextFlow: Towards Better Understanding of Evolving Topics in Text''\cite{cui_textflow:_2011}.
We choose \textit{TextFlow} based on two considerations. First, it's not too difficult for a novice but still a novel design that requires extra effect to clarify it.
Second, it is a typical abstract data visualization that is fully consist of graphical element, which is in the coverage of our edge detection algorithm. 

This visualization design employs three visual units to demonstrate the evolving topics in text data. It conveys multiple level results of topic evolution analysis: a set of topics
with splitting/merging relationships among each other, which encodes a series of topic flows, a set of critical events, which encodes glyphs, and the keyword correlations, which encode threads.  

\subsection{Procedure}
We run a two-hour long sessions, which is consist of 3 phases: (1)\textit{learn visualization}, (2)\textit{idea generation and sketch}, (3)\textit{authoring}.\par
In the \textit{learn visualization} phase, participants read the literature description we extract from the paper, which is two-page long and describes the visual design with diagrams. This phase ends when the participants report the experimenters that they finished reading and understund this visual design. 
This phase takes about 15min, since all the participants are experts in data visualization and familiar with reading such papers.

In the \textit{idea generation and sketch} phase, participants are asked to sketch ideas for introducing \textit{TextFlow} to general public. They are encouraged to give considerations to (i) convey the insight to the general public; (ii) organize a clear narrative structure; (iii) think about additional annotation required. Participants are asked to think aloud and experimenters are present in the room to observe. 

In the \textit{authoring} phase, participants implement the ideas in their sketch as detailed as possible in a one-hour-long session. Participants in control group use Power Point, a presentation making tool that all the participants are familiar with. In experimental group, before authoring, experimenters demonstrate the capacity of Narvis through an automatic step by step tutorial included in Narvis. This training lasts about 15 min and is not counted in the one-hour authoring session. Participants are also allowed to ask additional questions in the authoring phase.

\subsection{Results}
We analyzed the following material: 1) video and notes that the experimenters took during the user study session, which the participants consented to. 2) the slides and the sketch created by participants, 3) the interview with the editor participants, 4) the ranking, comments, answers, click stream data from the audience participants. 
\subsubsection{Authoring Experience of Narvis}

\noindent
\textbf{Authoring time}

\noindent
\textbf{Ideation}

\subsubsection{Generated Slideshow}

\noindent
\textbf{Number of produced slides}

\noindent
\textbf{The utility of attention cue}

\noindent
\textbf{Information coverage}

\noindent
\textbf{Information delivery}

\noindent
\textbf{Subjective rating}
We conducted a first pilot study to ensure the clarity of the instructions and control the time of experiments. 

We asked a group of 20 volunteers to evaluate the quality of the generated slideshow from 4 aspects, reablibity, . We conducted a questionnaire in advance to make sure that they all have no experience or knowledge in advanced data visualization. In a one-hour session, they are asked to view, comment, and rate these slide shows. They also answer a series of questions to check their understanding of the visualization design. \par 
We record video during this session with the participants permission. For participants who review the slide show generated by Narvis, their click activity will be recorded automatically and they can make comments on the slides. These click stream data, as well as the comments stream, will be used to generate a report, which will then send to its editor. \par
To conclude the user study, the experimenters conduct an interview with the participants about their authoring experience, the issues they encountered, if there are any, and the feedback report Narvis generated. \par
\noindent{}
\textbf{Subjective Assessments of produced slideshows}

%\subsubsection{}
%\subsubsection{}
%xuke\par
%reading 15min\par
%draft 5min \par
%making slides 40min \par
%qiaomu\par
%reading 14min\par
%draft 5min\par
%making slide 40min\par
%\subsubsection{Generated slideshow}
%\subsubsection{Authoring experience}