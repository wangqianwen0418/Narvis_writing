
\section{Introduction} %for journal use above \firstsection{..} instead
For data with complicated structure, simple data visualizations like bar chart and pie chart maybe unsatisfying for a comprehensive display. By introducing metaphors borrowed from nature ~\cite{cao_whisper:_2012,huron_visual_2013}, applying carefully designed layout algorithms~\cite{wu_opinionflow:_2014,chi_morphable_2015}, and sophisticatedly combining existing visualizations~\cite{zhao_x0023;fluxflow:_2014}, novel visual presentations help people identify patterns, trends and correlations hidden in data. However, these advanced visualizations are usually not intuitively recognizable. Users need to go through some training, for example, reading a long and dry literal description, before they grasp the knowledge required to understand and freely explore a visualization.\par
What is more, even people inside our community can suffer when they are required to introduce a visualization design, especially when the visual grammars have complicated logic dependency, or when their audiences have little prior knowledge about visualization techniques.\par
As a result, these advanced visualization technologies, in spite of
the fact that their utility has been verified by domain experts from various fields, haven't been widely exposed to the general audiences. Mainstream media is still dominated by these well-established yet simple visualizations, such as bar charts, pie charts and so on.

For a visualization, its core design space can be described as the orthogonal combination of two aspects: graphical elements and visual channels that control their appearance~\cite{munzner_visualization_2014}. But why the explanation of these two things is so complicated? 

This problem mainly arises from the fact that advanced visualization designs usually attempt to deliver a great amount of information. First, it would overload an audience if we inundate them with all the information at one time. Second, even if we try to explain it sequentially, considering the logic dependence existing, an improper explanation might confuse the audience. For example, in a node-link diagram, a node should be introduced before the links connecting it. In an advanced visualization design, which has more components than just nodes and links, it is challenging to identify a proper explaining order. Third, when digesting such a considerable amount of information, audiences can easily get distracted or forget previous information.   

Thus, to better introduce an advanced visualization, we should convey its information sequentially and in a specific order. Attention guidance and reminders are also needed to make sure that audiences are following this order, not getting distracted or forgetting previous information.

Narrative, which means ``connected events presented in a sequence'', has long been used to share complex information~\cite{schmidt_living_2017}. As the data visualization field is maturing, many researchers have moved their focus from analysis to presentation, making narrative data visualization an emerging topic~\cite{kosara_storytelling:_2013}. Many efforts have been
made to define, classify, and provide design suggestions for narrative data visualization~\cite{segel_narrative_2010,hullman_deeper_2013,gershon_what_2001}. Some visualization systems have already incorporated narrative modules into their design~\cite{eccles_stories_2007,bryan_temporal_2016}. However, most work is focused on communicating the conclusion of data analysis, and there is few discussion about how to guide the audiences to learn a visual design. 

Here, we present a framework to introduce new visualization designs. Based on our analysis of the structure, logic dependency, and visual distractions existing in a visualization design, we develop an authoring tool to decompose a visualization, reorganize extracted visual elements, and explain their visual grammars one by one through animated transition in the form of a slideshow. Through incorporating a narrative sequence, appropriate chunks of information, rather than all the information, are delivered to the audience at one time, effectively avoiding information overload. Visual attention guidance, such as flickering, highlighting, and morphing are used to lead audiences' attention to newly added information. We explore this framework on text visualizations, which is a typical visualization and widely applied. But we believe our work generalizes to other kinds of visualizations. 

To the best of our knowledge, this is the first attempt to introducing a visual design in a constructing way. Our contributions are as below: 1) A paradigm for decomposing visualizations. It analyzes the hierarchical structure of its components, the relationships between components, and visual distraction existing. 2) A framework for explaining a visual design, which is the result of consulting theory from graphical perception process, techniques in narrative visualization, various attention cues in animation, and empirical observations of numerous visualization designs. 3) An authoring tool to generate and edit the narrative visual encoding explanation.
 We conjecture our work can motivate and enable people to use more advanced visualization designs, supporting the democratization of data visualization.
 