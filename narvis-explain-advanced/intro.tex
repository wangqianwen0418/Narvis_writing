
\section{Introduction} %for journal use above \firstsection{..} instead
% For data with complicated structure, simple data visualizations like bar chart and pie chart maybe unsatisfying for a comprehensive display. 
Simple data visualizations, such as bar chart and line chart, are not sufficient to meet various needs of end users.
By introducing metaphors borrowed from nature ~\cite{cao_whisper:_2012,huron_visual_2013}, applying carefully designed layout algorithms~\cite{wu_opinionflow:_2014,chi_morphable_2015}, and sophisticatedly combining existing visualizations~\cite{zhao_x0023;fluxflow:_2014}, novel visual presentations help users identify patterns, trends and correlations in data. However, these novel visualizations are sometimes complex and require steep learning curve, which hinder users from grasping insights into them.
% not intuitively recognizable. Users need to go through some training, for example, reading a long and dry literal description, before they grasp the knowledge required to understand and freely explore a visualization.\par

As the field of data visualization is getting mature, many researchers have moved their focus from analysis to presentation.
From the perspective of visualization experts, they are willing to generalize novel designs to other domains, and make their designs visible to general audience. However, they face three challenges when introducing novel yet complex visual designs to general audience.
% This problem mainly arises from the fact that advanced visual designs usually attempt to deliver a great amount of information. 
First, complex visual designs attempt to deliver a great amount of information, which would overload the audience if all information is dumped at one time. Second, even if experts try to explain complex visual designs with an order. Considering the logic dependence existing, an improper explanation might confuse the audience. For example, in a node-link diagram, a node should be introduced before the links connecting it. In a complex visual design, which has more components than just nodes and links. To identify a proper logic for explanation is challenging. Third, given considerable amount of information, the audience can easily get lost or forget previous information. How to engage the attention and attract their attention is challenging.  

% , who have little prior knowledge about visualization techniques. 
% What is more, even people inside our community can suffer when they are required to introduce a visual design, especially when the visual grammars have complicated logic dependency, or when their audiences have little prior knowledge about visualization techniques.\par
% As a result, these advanced visualization technologies, in spite of
% the fact that their utility has been verified by domain experts from various fields, haven't been widely exposed to the general audiences. Mainstream media is still dominated by these well-established yet simple visualizations, such as bar charts, pie charts and so on.
% For a visualization, its core design space can be described as the orthogonal combination of two aspects: graphical elements and visual channels that control their appearance~\cite{munzner_visualization_2014}. But why the explanation of these two things is so complicated? 


% To better introduce an advanced visualization, we should convey its information sequentially and in a specific order. Attention guidance and reminders are also needed to make sure that audiences are following this order, not getting distracted or forgetting previous information.

Narrative data visualization is becoming an emerging topic~\cite{kosara_storytelling:_2013}. 
Narrative, which means ``connected events presented in a sequence'', has long been used to share complex information~\cite{schmidt_living_2017}. 
Many efforts have been made to define, classify, and provide design suggestions for narrative data visualization~\cite{segel_narrative_2010,hullman_deeper_2013,gershon_what_2001}. Some visualization systems have already incorporated narrative modules into their design~\cite{eccles_stories_2007,bryan_temporal_2016}. However, most prior arts focus on communicating the conclusion of data analysis, and few discussions aim at how to guide the audiences to learn a visual design. 

In this paper, we present a framework to introduce new visual designs. Based on our analysis of the structure, logic dependency, and visual distractions existing in a visual design, we develop an authoring tool, Narvis, to decompose a visualization, reorganize extracted visual elements, and explain their visual grammars one by one through animated transition in the form of a slideshow. Through incorporating a narrative sequence, appropriate chunks of information, rather than all the information, are delivered to the audience at one time, which avoids information overload. Visual attention guidance, such as flickering, highlighting, and morphing are used to lead audiences' attention to newly added information. We ground this framework on text visualizations, which are widely applied to designed to explore and analyze textural data. But we believe our work generalizes to other kinds of visualizations. 

To the best of our knowledge, this is the first attempt to introducing a visual design in a constructing way. Our contributions are as below: 1) A paradigm for decomposing visualizations. It analyzes the hierarchical structure of its components, the relationships between components, and possible visual distractions. 2) A framework for explaining a visual design, which is the result of consulting theory from graphical perception process, techniques in narrative visualization, various attention cues in animation, and empirical observations of numerous visual designs. 3) An authoring tool to generate and edit the narrative visual encoding explanation.
 We conjecture our work can motivate and enable people to use more advanced visual designs, supporting the democratization of data visualization.
 


