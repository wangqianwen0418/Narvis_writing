
\section{Introduction} %for journal use above \firstsection{..} instead
For data with complicated structure, naive data visualization like bar chart and pie chart maybe unsatisfying for a comprehensive display. By introducing metaphors borrowed from nature \cite{cao_whisper:_2012,huron_visual_2013}, applying carefully designed layout algorithms\cite{wu_opinionflow:_2014,chi_morphable_2015}, and sophisticatedly combining existing visualizations\cite{zhao_x0023;fluxflow:_2014}, novel visual presentations help people identify patterns, trends and correlations hidden in data. However, these advanced visualizations are usually not intuitively recognizable. Users need to go through some training, for example, reading a long and boring literal description, before they grasp the knowledge required to understand and freely explore a visualization.\par
What is more, even designers of these advanced visualizations suffer when they are required to introduce their design, especially when the visual encoding has complicated logic dependency, or when their audience have little prior knowledge about visualization techniques.\par
As a result, these advanced visualization technologies, in spite of
the fact that their utility has been verified by domain experts from various fields, gain little exposure outside the visual community. Main stream media is still dominated by naive visualizations, such as bar charts, pie charts and so on.

For a visualization, its core design space can be described as the orthogonal combination of two aspects: graphical elements called marks and visual channels to control their appearance\cite{munzner_visualization_2014}. But why the explanation of these two things is so complicated? 

This problem mainly arises from the fact that advanced visualization designs usually attempt to delivery a great amount of information. First, it would overload an audience if we inundated them with all the information at one time. Second, even if we tried to explain it sequentially, considering the logic dependency existing among visual elements, an improper explanation could totally confuse the audience. For example, in a node link diagram, a node should be introduced before the links connecting it. In an advanced visualization design, which has more components than just nodes and links, it is challenging to identify a proper explaning oder. Third, when digesting such a considerable amount of information, audiences can easily get distracted or forget previous information.   

Thus, to better introduce an advanced visualization, we should convey its information sequentically and in a specific order. Attention guidance and reminders are also needed to make sure that audiences are following this order, not getting distracted or forgetting previous information.

Narrative, which means “connected events presented in a sequence”, has long been used to share complex information. \cite{schmidt_living_2017}As the data visualization field is maturing, many researchers have moved their focus from analysis to presentation, making narrative data visualization an emerging topic\cite{kosara_storytelling:_2013}. Many efforts have been
made to define, classify, and provide design suggestions for narrative data visualization\cite{segel_narrative_2010,hullman_deeper_2013,gershon_what_2001}. Some visualization systems have already incorporated narrative modules into their design\cite{eccles_stories_2007,bryan_temporal_2016}. However, current work is mainly focused on communicating the conclusion of analyses, rather than guiding the audience how to read a visualization. 

Here, we present a prototype to introduce new visualization design. Based on our analysis of the structure, logic dependency, and visual distraction existing in a visualization design, we develop an authoring tool to decompose a visualization, reorganize extracted visual elements, and explain their visual encodings one by one through animated transition in the form of slideshow. Through incorporating a narrative sequence, appropriate chunks of information, rather than all the information, is delivered to the audience at one time, effectively avoiding information overload. Reminders, such as questions, summarizations and repetitions are woven into the narrative sequence to enhance the audience’s memory while visual attention guidance, such as flickering, highlighting, and morphing are used to lead their attention to newly added information. (字数超了就删掉)

To the best of our knowledge, this is the first attempt to explain visual encoding with narrative. Our contributions are as below: 1). A paradigm for decomposing visualizations. It analyzes the hierarchical structure of its components, the relationships between components, and visual distraction existing. 2) A framework for explaining visualization design, which is the result of consulting theory from graphical perception process, techniques in narrative visualization, various attention cues in animation, and empirical observations of numerous visualization designs. 3) An authoring tool to generate and edit the narrative visual encoding explanation
 We conjecture our work can motivate and enable people to use more advanced visualization designs, supporting the democratization of data visualization.
 