\section {Related Work}
In this section, we provide an overview of prior research around the analysis of narrative structure in data visualization, animation in data visualization, and existing authoring tools for narrative visualizations.

\subsection{Structure of Narrative Data Visualization}
Narrative is as old as human history\cite{cunningham_culture_2009}.  People in the fields of literature, comics \cite{cohn_visual_2013} and cinema \cite{schmidt_living_2017} have gone to great lengths to analyze the sequencing and forms of grouping used in a narrative, as well as how they affect the meaning a narrative tries to deliver. 

Some people believe that work from other fields can inspire researchers in the visual data community. Amini et al.\cite{amini_understanding_2015} borrow concepts from comics \cite{cohn_visual_2013} to classify and analyze the structure of data videos. Wang et al. \cite{wang_animated_2016} adopt two representative tactics, time-remapping and foreshadowing, from cinematographers to organize a narrative sequence for visualizing temporal data. 

Other researchers, on the other side, focus on the narrative structures exclusively for data visualization. 
Satyanarayan and Heer, through interviews with professional journalists\cite{satyanarayan_authoring_2014}, define the core abstractions of narrative data visualization as state-based scenes, visualization parameters, dynamic graphical and textual annotations, and interaction triggers. Hullman et al.\cite{hullman_deeper_2013}, by identifying the change in data attributes, propose a graph-driven approach to automatically identify effective narrative sequences for linearly presenting a set of visualizations. 

These works, however, rarely discuss the narrative structures used for explaining visual designs. We hope our work can fill this gap.
\subsection{Animation for Data Visualizations}
There is a wide discussion about the effects of animation when used in a data visualization environment.
Animation can facilitate the cognitive process. Heer and Robertson \cite{heer_animated_2007-1} confirm the effectiveness of animation when relating data visualizations backed by a shared dataset. Ruchikachorn et al.\cite{ruchikachorn_learning_2015}, going a step further, design morphing animations which bridge the gap between a familiar visualization and an unfamiliar one, thus introducing a new visualization design through animation. Graphdiaries \cite{bach_graphdiaries:_2014} uses animation to help audiences track and understand changes in a dynamic visualization. 

Animation can also be an effective tool to attract and guide visual attention. Huber et al. \cite{huber_visualizing_2005} study the perceptual properties of different kinds of animation, as well as their effects on human attention. Waldner et al. \cite{waldner_attractive_2014} focus on a specific animation: flicker. By dividing the animation into an “orientation stage” and an “engagement stage”, they strike a good balance between the attraction effectiveness and annoyance caused by flickering. 

It is, however, noteworthy that animation, in spite of all the advantages mentioned above, can bring about negative effects when used improperly\cite{robertson_effectiveness_2008}. Our work is based on the results of previous research, which provides a guideline on how to implement animations in our system.

\subsection{Authoring Tools for Narrative Visualizations}
The extensive needs of data communication exist not only in the data visualization field but also in journalism, media, and so on. This has motivated researchers to investigate ways for authoring narrative visualization. 

User experience is of great concern when utilizing an authoring tool. Sketch story \cite{lee_sketchstory:_2013}, with its freeform sketch interaction, provides a more engaging way to create and present narrative visualization. Dataclips \cite{amini_authoring_2017} lowers the barrier of crafting a narrative visualization by providing a library of data clips, allowing non-experts to be involved in the production of narrative visualization. 

However, it is information delivery that is the core consideration of an authoring tool. Existing authoring tools usually choose a specific type of narrative visualization based on the information type \cite{amini_authoring_2017, fulda_timelinecurator:_2016}. Meanwhile, integrating an authoring tool for narrative visualization with a  data analysis tool has become a trend since it effectively bridges the gap between data analysis and data communication\cite{eccles_stories_2007, bryan_temporal_2016,lee_more_2015}. 
 
These tools offer inspiring user interaction design as well as good examples to implement narrative visualization. However, they treat visual encodings as cognitively obvious attributes that can be universally recognized without a formal introduction, making them inapplicable in our case. 

\subsection{Decompose a Data Visualization}
Clarifying the design space of a data visualization can help people get a better understanding of how it is constructed. Tamara \cite{munzner_visualization_2014} proposes that it ``can be described as an orthogonal combination of two aspects: graphical elements called marks and visual channels to control their appearance''. Borrowing the concept of physical building blocks such as Lego, Huron et al. \cite{huron_constructive_2014} extends the design space of a data visualization, defining the components of a data visualization as a token, token grammar, environment and assembly model.

Such theoretical work motivates the designers of visualization tools to contribute efficient high-level visualization systems rather than low-level graphical systems\cite{bostock_protovis:_2009,mendez_ivolver:_2016}. 

On the other hand, theoretically identifying the basic components of a data visualization enables people to physically extract them, and remap them to an alternative design without involving any programming work. Harper and Agrawala \cite{harper_deconstructing_2014} contribute a tool that extracts visual variables from existing D3 visualization designs to generate a new design.Huang et al.\cite{Huang:2007:SUI:1284420.1284427} propose a system that recognizes and interprets imaged
infographics from a scanned document. Revision\cite{savva_revision:_2011} applies computer vision methods to recognize the types, marks, encodings of a data visualization, and allows the users to create a new design based on these data. 

However, these decomposing methods exclusively focus on simple visualization designs, such as bar chart, line chart, dot chart, and are not applicable for advanced visualization designs, which assemble miscellaneous visualization approaches to realize a novel presentation. Moreover, these methods are not specified of explaining a visual design, thus giving no consideration for graphical perception process and visual attention shift.

