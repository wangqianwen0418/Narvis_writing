%\documentclass[journal]{vgtc}                % final (journal style)
\documentclass[review,journal]{vgtc}         % review (journal style)
%\documentclass[widereview]{vgtc}             % wide-spaced review
%\documentclass[preprint,journal]{vgtc}       % preprint (journal style)

%% Uncomment one of the lines above depending on where your paper is
%% in the conference process. ``review'' and ``widereview'' are for review
%% submission, ``preprint'' is for pre-publication, and the final version
%% doesn't use a specific qualifier.

%% Please use one of the ``review'' options in combination with the
%% assigned online id (see below) ONLY if your paper uses a double blind
%% review process. Some conferences, like IEEE Vis and InfoVis, have NOT
%% in the past.

%% Please note that the use of figures other than the optional teaser is not permitted on the first page
%% of the journal version.  Figures should begin on the second page and be
%% in CMYK or Grey scale format, otherwise, colour shifting may occur
%% during the printing process.  Papers submitted with figures other than the optional teaser on the
%% first page will be refused. Also, the teaser figure should only have the
%% width of the abstract as the template enforces it.

%% These few lines make a distinction between latex and pdflatex calls and they
%% bring in essential packages for graphics and font handling.
%% Note that due to the \DeclareGraphicsExtensions{} call it is no longer necessary
%% to provide the the path and extension of a graphics file:
%% \includegraphics{diamondrule} is completely sufficient.
%%
\ifpdf%                                % if we use pdflatex
  \pdfoutput=1\relax                   % create PDFs from pdfLaTeX
  \pdfcompresslevel=9                  % PDF Compression
  \pdfoptionpdfminorversion=7          % create PDF 1.7
  \ExecuteOptions{pdftex}
  \usepackage{graphicx}                % allow us to embed graphics files
  \DeclareGraphicsExtensions{.pdf,.png,.jpg,.jpeg} % for pdflatex we expect .pdf, .png, or .jpg files
\else%                                 % else we use pure latex
  \ExecuteOptions{dvips}
  \usepackage{graphicx}  % allow us to embed graphics files
  \DeclareGraphicsExtensions{.eps}     % for pure latex we expect eps files
\fi%

%% it is recomended to use ``\autoref{sec:bla}'' instead of ``Fig.~\ref{sec:bla}''
\graphicspath{{figures/}{pictures/}{images/}{./}} % where to search for the images

\usepackage{microtype}                 % use micro-typography (slightly more compact, better to read)
\PassOptionsToPackage{warn}{textcomp}  % to address font issues with \textrightarrow
\usepackage{textcomp}                  % use better special symbols
\usepackage{mathptmx}                  % use matching math font
\usepackage{colortbl}
\usepackage{dingbat}
\usepackage{times}                     % we use Times as the main font
\renewcommand*\ttdefault{txtt}         % a nicer typewriter font
\usepackage{cite}                      % needed to automatically sort the references
\usepackage{tabu}                      % only used for the table example
\usepackage{booktabs}                  % only used for the table example
\usepackage{CJK}  
%\usepackage[top=2cm, bottom=2cm, left=2cm, right=2cm]{geometry}  
\usepackage{algorithm}  
\usepackage{algorithmicx}  
\usepackage{algpseudocode}  
\usepackage{amsmath}  
\renewcommand{\algorithmicrequire}{\textbf{Data:}}
\renewcommand{\algorithmicensure}{\textbf{Result:}}
\renewcommand{\algorithmicforall}{\textbf{foreach:}}
%\floatname{algorithm}{}

\newcommand{\siwei}[1]{\textcolor{red}{#1}}
\newcommand{\original}[1]{\textcolor{blue}{#1}}
\newcommand{\notsure}[1]{\textcolor{blue}{#1}}


%% We encourage the use of mathptmx for consistent usage of times font
%% throughout the proceedings. However, if you encounter conflicts
%% with other math-related packages, you may want to disable it.

%% In preprint mode you may define your own headline.
%\preprinttext{To appear in IEEE Transactions on Visualization and Computer Graphics.}

%% If you are submitting a paper to a conference for review with a double
%% blind reviewing process, please replace the value ``0'' below with your
%% OnlineID. Otherwise, you may safely leave it at ``0''.
\onlineid{162}

%% declare the category of your paper, only shown in review mode
\vgtccategory{Application}
%% please declare the paper type of your paper to help reviewers, only shown in review mode
%% choices:
%% * algorithm/technique
%% * application/design study
%% * evaluation
%% * system
%% * theory/model
\vgtcpapertype{application/design study}

%% Paper title.
%\title{Narvis: How to Explain An Advanced Visualization Design}
\title{Narvis: Authoring Narrative SlideShows for Presenting \\Data Visual Designs in A Constructing Way}
%% This is how authors are specified in the journal style

%% indicate IEEE Member or Student Member in form indicated below
\author{Qianwen Wang, Zhen Li, Siwei Fu, and Huamin Qu}
\authorfooter{
%% insert punctuation at end of each item
\item
 Roy G. Biv is with Starbucks Research. E-mail: roy.g.biv@aol.com.
\item
 Ed Grimley is with Grimley Widgets, Inc.. E-mail: ed.grimley@aol.com.
\item
 Martha Stewart is with Martha Stewart Enterprises at Microsoft
 Research. E-mail: martha.stewart@marthastewart.com.
}

%other entries to be set up for journal
\shortauthortitle{Biv \MakeLowercase{\textit{et al.}}: Global Illumination for Fun and Profit}
%\shortauthortitle{Firstauthor \MakeLowercase{\textit{et al.}}: Paper Title}

%% Abstract section.
\abstract{
Visual designs can be complex in modern data visualization systems, which poses special challenges for explaining them to the general audience. However, there is few theoretical work or presentation tool tailored for introducing a data visualization design. In this study, we present Narvis, an authoring tool for the crafting of  narrative slideshows that introduces a visualization design. In Narvis, a visualization is specified as a combination of visual units and demonstrated in a constructing way. To better guide the crafting of an introduction slideshow, we incorporate lessons from previous work with our observation and propose a hierarchical constructing model, which consist of: conceptual components at different hierarchical levels, the process that components assemble another component at a higher level, and suggestions for the utility of narratives when introducing different components. Guided by this model, we implement a library of templates in Narvis. It enables the editors crafting an introduction slideshow through assembling these templates, thus achieves a level of expressiveness while improving efficiency. Narvis is designed and implemented specified for text visualizations but can be generalized to other types of visualizations. We evaluate Narvis through a preliminary evaluation of the authoring experience, a quantitative analysis of the generated slide show, and a qualitative analysis of the generated slideshow in the aspect of aesthetic, engagement, readability and utility. 

} 
% end of abstract

%% Keywords that describe your work. Will show as 'Index Terms' in journaln
%% please capitalize first letter and insert punctuation after last keyword
\keywords{User Interface, Visualization System and Toolkit Design, }

%% ACM Computing Classification System (CCS). 
%% See <http://www.acm.org/class/1998/> for details.
%% The ``\CCScat'' command takes four arguments.

\CCScatlist{ % not used in journal version
 \CCScat{K.6.1}{Management of Computing and Information Systems}%
{Project and People Management}{Life Cycle};
 \CCScat{K.7.m}{The Computing Profession}{Miscellaneous}{Ethics}
}

%% Uncomment below to include a teaser figure.
\teaser{
  \centering
  \includegraphics[width=\linewidth]{teaser}
  \caption{Example of an introduction slide show of \textit{TextFlow}\cite{cui_textflow:_2011} generated by an expert in data visualization using Narvis. This slideshow consist of (a) a cover, (b) a decomposing animation, (c) introducing the design in a constructing way.  }
	\label{fig:teaser}
}

%% Uncomment below to disable the manuscript note
%\renewcommand{\manuscriptnotetxt}{}

%% Copyright space is enabled by default as required by guidelines.
%% It is disabled by the 'review' option or via the following command:
% \nocopyrightspace

\vgtcinsertpkg

%%%%%%%%%%%%%%%%%%%%%%%%%%%%%%%%%%%%%%%%%%%%%%%%%%%%%%%%%%%%%%%%
%%%%%%%%%%%%%%%%%%%%%% START OF THE PAPER %%%%%%%%%%%%%%%%%%%%%%
%%%%%%%%%%%%%%%%%%%%%%%%%%%%%%%%%%%%%%%%%%%%%%%%%%%%%%%%%%%%%%%%%
\begin{document} 
\maketitle

\section{Introduction} %for journal use above \firstsection{..} instead
For data with complicated structure, simple data visualizations like bar chart and pie chart maybe unsatisfying for a comprehensive display. By introducing metaphors borrowed from nature ~\cite{cao_whisper:_2012,huron_visual_2013}, applying carefully designed layout algorithms~\cite{wu_opinionflow:_2014,chi_morphable_2015}, and sophisticatedly combining existing visualizations~\cite{zhao_x0023;fluxflow:_2014}, novel visual presentations help people identify patterns, trends and correlations hidden in data. However, these advanced visualizations are usually not intuitively recognizable. Users need to go through some training, for example, reading a long and dry literal description, before they grasp the knowledge required to understand and freely explore a visualization.\par
What is more, even people inside our community can suffer when they are required to introduce a visualization design, especially when the visual grammars have complicated logic dependency, or when their audiences have little prior knowledge about visualization techniques.\par
As a result, these advanced visualization technologies, in spite of
the fact that their utility has been verified by domain experts from various fields, haven't been widely exposed to the general audiences. Main stream media is still dominated by these well established yet simple visualizations, such as bar charts, pie charts and so on.

For a visualization, its core design space can be described as the orthogonal combination of two aspects: graphical elements and visual channels that control their appearance~\cite{munzner_visualization_2014}. But why the explanation of these two things is so complicated? 

This problem mainly arises from the fact that advanced visualization designs usually attempt to delivery a great amount of information. First, it would overload an audience if we inundate them with all the information at one time. Second, even if we try to explain it sequentially, considering the logic dependence existing, an improper explanation might confuse the audience. For example, in a node link diagram, a node should be introduced before the links connecting it. In an advanced visualization design, which has more components than just nodes and links, it is challenging to identify a proper explaining oder. Third, when digesting such a considerable amount of information, audiences can easily get distracted or forget previous information.   

Thus, to better introduce an advanced visualization, we should convey its information sequentially and in a specific order. Attention guidance and reminders are also needed to make sure that audiences are following this order, not getting distracted or forgetting previous information.

Narrative, which means``connected events presented in a sequence'', has long been used to share complex information~\cite{schmidt_living_2017}. As the data visualization field is maturing, many researchers have moved their focus from analysis to presentation, making narrative data visualization an emerging topic~\cite{kosara_storytelling:_2013}. Many efforts have been
made to define, classify, and provide design suggestions for narrative data visualization~\cite{segel_narrative_2010,hullman_deeper_2013,gershon_what_2001}. Some visualization systems have already incorporated narrative modules into their design~\cite{eccles_stories_2007,bryan_temporal_2016}. However, most work is focused on communicating the conclusion of data analysis, and there is few discussion about how to guide the audiences to learn a visual design. 

Here, we present a framework to introduce new visualization designs. Based on our analysis of the structure, logic dependency, and visual distractions existing in a visualization design, we develop an authoring tool to decompose a visualization, reorganize extracted visual elements, and explain their visual grammars one by one through animated transition in the form of a slideshow. Through incorporating a narrative sequence, appropriate chunks of information, rather than all the information, are delivered to the audience at one time, effectively avoiding information overload. Visual attention guidance, such as flickering, highlighting, and morphing are used to lead audiences' attention to newly added information. We explore this framework on text visualizations, which is a typical visualization and widely applied. But we believe our work generalizes to other kinds of visualizations. 

To the best of our knowledge, this is the first attempt to introducing a visual design in a constructing way. Our contributions are as below: 1) A paradigm for decomposing visualizations. It analyzes the hierarchical structure of its components, the relationships between components, and visual distraction existing. 2) A framework for explaining a visual design, which is the result of consulting theory from graphical perception process, techniques in narrative visualization, various attention cues in animation, and empirical observations of numerous visualization designs. 3) An authoring tool to generate and edit the narrative visual encoding explanation.
 We conjecture our work can motivate and enable people to use more advanced visualization designs, supporting the democratization of data visualization.
 
\section {Related work}
In this section, we provide an overview of prior research around the analysis of narrative structure in data visualization, animation in data visualization, and existing authoring tools for narrative visualizations.

\subsection{Structure of Narrative Data Visualization}
Narrative is as old as human history\cite{cunningham_culture_2009}.  People in the fields of literature, comics \cite{cohn_visual_2013} and cinema \cite{schmidt_living_2017} have gone to great lengths to analyze the sequencing and forms of grouping used in a narrative, as well as how they affect the meaning a narrative tries to deliver. 

Some people believe that work from other fields can inspire researchers in the visual data community. Amini et al.\cite{amini_understanding_2015} borrow concepts from comics \cite{cohn_visual_2013} to classify and analyze the structure of data videos. Wang et al. \cite{wang_animated_2016} adopt two representative tactics, time-remapping and foreshadowing, from cinematographers to organize a narrative sequence for visualizing temporal data. 

Other researchers, on the other side, focus on the narrative structures exclusively for data visualization. 
Satyanarayan and Heer, through interviews with professional journalists\cite{satyanarayan_authoring_2014}, define the core abstractions of narrative data visualization as state-based scenes, visualization parameters, dynamic graphical and textual annotations, and interaction triggers. Hullman et al.\cite{hullman_deeper_2013}, by identifying the change in data attributes, propose a graph-driven approach to automatically identify effective narrative sequences for linearly presenting a set of visualizations. 

These works, however, rarely discuss the narrative structures used for explaining visual designs. We hope our work can fill this gap.
\subsection{Animation for data visualization}
There is a wide discussion about the effects of animation when used in a data visualization environment.
Animation can facilitate the cognitive process. Heer and Robertson \cite{heer_animated_2007-1} confirm the effectiveness of animation when relating data visualizations backed by a shared dataset. Ruchikachorn et al.\cite{ruchikachorn_learning_2015}, going a step further, design morphing animations which bridge the gap between a familiar visualization and an unfamiliar one, thus introducing a new visualization design through animation. Graphdiaries \cite{bach_graphdiaries:_2014} uses animation to help audiences track and understand changes in a dynamic visualization. 

Animation can also be an effective tool to attract and guide visual attention. Huber et al. \cite{huber_visualizing_2005} study the perceptual properties of different kinds of animation, as well as their effects on human attention. Waldner et al. \cite{waldner_attractive_2014} focus on a specific animation: flicker. By dividing the animation into an “orientation stage” and an “engagement stage”, they strike a good balance between the attraction effectiveness and annoyance caused by flickering. 

It is, however, noteworthy that animation, in spite of all the advantages mentioned above, can bring about negative effects when used improperly\cite{robertson_effectiveness_2008}. Our work is based on the results of previous research, which provide a guideline on how to implement animations in our system.

\subsection{Authoring tools for narrative visualization}
The extensive needs of data communication exist not only in the data visualization field but also in journalism, media, and so on. This has motivated researchers to investigate ways for authoring narrative visualization. 

User experience is of great concern when utilizing an authoring tool. Sketch story \cite{lee_sketchstory:_2013}, with its freeform sketch interaction, provides a more engaging way to create and present narrative visualization. Dataclips \cite{amini_authoring_2017} lowers the barrier of crafting a narrative visualization by providing a library of data clips, allowing non-experts to be involved in the production of narrative visualization. 

However, it is information delivery that is the core consideration of an authoring tool. Existing authoring tools usually choose a specific type of narrative visualization based on the information type \cite{amini_authoring_2017, fulda_timelinecurator:_2016}. Meanwhile, integrating an authoring tool for narrative visualization with a  data analysis tool has become a trend since it effectively bridges the gap between data analysis and data communication\cite{eccles_stories_2007, bryan_temporal_2016,lee_more_2015}. 
 
These tools offer inspiring user interaction design as well as good examples to implement narrative visualization. However, they treat visual encodings as cognitively obvious attributes that can be universally recognized without a formal introduction, making them inapplicable in our case. 

\subsection{Decompose a data visualization}
Clarifying the design space of a data visualization can help people get a better understanding of how it is constructed. Tamara \cite{munzner_visualization_2014} proposes that it ``can be described as an orthogonal combination of two aspects: graphical elements called marks and visual channels to control their appearance''. Borrowing the concept of physical building blocks such as Lego, Huron et al. \cite{huron_constructive_2014} extends the design space of a data visualization, defining the components of a data visualization as a token, token grammar, environment and assembly model.

Such theoretical work motivates the designers of visualization tools to contribute efficient high-level visualization systems rather than low-level graphical systems\cite{bostock_protovis:_2009,mendez_ivolver:_2016}. 

On the other hand, theoretically identifying the basic components of a data visualization enables people to physically extract them, and remap them to an alternative design without involving any programming work. Harper and Agrawala \cite{harper_deconstructing_2014} contribute a tool that extracts visual variables from existing D3 visualization designs to generate a new design.Huang et al.\cite{Huang:2007:SUI:1284420.1284427} propose a system that recognizes and interprets imaged
infographics from a scanned document. Revision\cite{savva_revision:_2011} applies computer vision methods to recognize the types, marks, encodings of a data visualization, and allows the users to create a new design based on these data. 

However, these decomposing methods exclusively focus on simple visualization designs, such as bar chart, line chart, dot chart, and are not applicable for advanced visualization designs, which assemble miscellaneous visualization approaches to realize a novel presentation. Moreover, these methods are not specified of explaining a visual design, thus giving no consideration for graphical perception process and visual attention shift.


\section{Introducing a data visualization} 
To help people better understand a data visualization design, we propose a method that introduces a data visualization through constructing, which has been proven as an effective teaching method\cite{huron_constructive_2014, chapman_constructive_1988} . Thus, there are three questions we need to answer: ``\textit{what are the basic components that compose a data visualization? }'' ,``\textit{what is the relationship between these components? }'', ``\textit{How should we deal with these relationships in our narrative?}''. At the same time, considering the large number of graphical elements employed in a data visualization design, how to orientate audience's attention to the focus should also be taken into consideration.

\subsection{Compositions of a Visualization}
We propose a model that decomposes a visualization into three levels of structure: visual primitives, visual units, and then an advanced visualization design. We apply this hierarchical structure theory to ``Opinionseer'' and decompose it, as shown in Fig.2. 

\textbf{A visual primitive} is one graphic element whose visual channels, such as color, width, height, are mapped to data attributes. We employ the definition in previous work\cite{huron_constructive_2014, satyanarayan_vega-lite:_2017}, use the term "grammar" to describe how the visual channels of a visual primitive is influenced by data. For instance, a point whose size and color are encoded is a visual primitive. How the two visual channels, size and color, are related to data attributes is its visual grammar. 

\textbf{A visual unit} is the assembly of visual primitives based on a certain construction rule, as tab.1 show. 
%For the taxonomy of constructing rule, we refer the ``alignment method'' in \cite{kucher2015text} but make it more specific. 
A visual primitive can assemble different visual units by following different constructing rule, as demonstrated in Fig.3. We are not pretending that our table includes all existing visual units, since new design is proposed constantly. A visual unit is the smallest functional unit of a visualization. Note that people might employ two visual primitives in a visualization design. For example, \textit{Visual Sedimentation}\cite{huron_visual_2013} employs two visual primitives, bar and dot, to construct a novel design. 


\textbf{A visualization} can be treated as the combination of visual units. An naive visualization can be as simple as one visual unit while an advanced one is usually the combination of several units. It doesn't simply put all visual units together but construct them with certain connections with each other, which is detailedly discussed in section 3.1.2.

\begin{figure}
 \centering % avoid the use of \begin{center}...\end{center} and use \centering instead (more compact)
 \includegraphics[width=\columnwidth]{hierarchic}
 \caption{An example of the hierarchical structure of a visualization, Opinion Seer\cite{wu_opinionseer:_2010}}
 \label{fig:hierarchic}
\end{figure}

\begin{figure}
\begin{minipage}{\columnwidth}
 \centering % avoid the use of \begin{center}...\end{center} and use \centering instead (more compact)
 \includegraphics[width=\columnwidth]{assemble}
\caption[assemble]
{ A dot, whose color and size are encoded, can assemble 
(a) a dot spiral chart
\protect\footnotemark{}
    , (b)a dot packing chart
  \footnotemark{}
  , and (c)a bubble chart
  \footnotemark{}
   by following different construction rules.
}
\end{minipage}
\label{fig:hierarchic}
\end{figure}
\footnotetext{https://www.pinterest.com/pin/16536723602037537/}
\footnotetext{https://plot.ly/~etpinard/84.embed}
\footnotetext{https://bl.ocks.org/mbostock/4063269}

\begin{table*}[tb]
  \caption{A taxonomy of visual units.\notsure{How to avoid the name ambiguities}}
  \label{tab:unit}
  \small
  \centering
  \begin{tabular}{|p{1.2cm}|p{1.2cm}|p{1.2cm}|p{1.2cm}|p{1.2cm}|p{1.2cm}|p{1.2cm}|p{1.2cm}|p{1.2cm}|}
  \toprule
   \textbf{} &\multicolumn{2}{|c|}{Polar Coordinates} &\multicolumn{3}{|c|}{Orthogonal Coordinates}&\multicolumn{3}{|c|}{Metric Dependent}   \\ 
  \midrule
  
 \textbf{} &\textbf{Radial} &\textbf{Spiral} &\textbf{Orthogonal} & \textbf{Parallel Align}&\textbf{Map}&\textbf{Cluster}&\textbf{Force-direct}&\textbf{Others}   \\ 
  \midrule
  \textbf{Dot} &    &Spiral Dot Chart&Scatter Plot, Bubble Chart & Dot Plot & Bubble Map &  Circle packing    &TopicPanorama\cite{7042494}  &    \\
  \midrule
  \textbf{Line}&  Radar Chart   &  Spiral Plot    &Node-link Diagram, Line Chart & Parallel Coordinates, Arc Diagram &    &   &     & \\ 
  \midrule
   \textbf{Flow}&  Chord Diagram   &    & &Parallel Sets, Sankey Diagram & 
   Flow Map  &   &   &\\
  \midrule
  \textbf{Area}&    &Area Spiral Chart &Stream Graph &  & & &   &\\ 
  \midrule
  \textbf{Bar}&      Radial Bar Chart & Spiral Bar Chart  & Candlestick Chart & Bar Chart  &    &    &    &\\
  \midrule
  \textbf{Cell}& Sunburst Diagram  &    & Matrix, Tree Map &     & & &   &\\
  \midrule
  \textbf{Wedge}& Pie Chart, Donut Chart &  &   &   &  &    &   &\\
  \midrule
  \textbf{Text}&    &Parallel Tag Cloud \cite{collins2009parallel} &    &  Sentence Tree  &     &Word Cloud  &   &    \\
  %\midrule
 % \textbf{Image}& & &Heatmap Matrix &Heatmap &\\
  \bottomrule
  
  \end{tabular}
  \vspace{1mm}
\end{table*}


\subsection{Relationship between compositions}
\subsubsection{Relationships between visual units}
An advanced visualization can be specified as the combination of several visual units. Through observing the approaches people apply to design new visualizations, we define four types of relationship between visual units: irrelevance, relevance, enhancement, and dependency. 

\textbf{Irrelevance} refers to that two visual units have no correlations in the visual grammar. It is a bi-directional relation. For example, 2 donut charts, Fig.4(a) and Fig.4(b), are applied to illustrate the distribution of age and gender groups respectively in a population. They are put together in Fig.4(c) just for space-efficiency and there is no correlation between these two charts. 

\textbf{Relevance} refers to that two visual units share some visual grammar and it is a bi-directional relation. For example, a line chart, Fig.4(d), and a bar chart, Fig.4(d), share the same encoding of horizontal position and they are put together in Fig.4(e). 
\notsure{According to our survey, color and position are the most commonly shared visual encodings, which might be the result that color and position usually encoded with simple while crucial information. }

\textbf{Enhancement} is an one-way relationship. If one visual unit ``A'' is the enhancement of another visual unit ``B'', it means that ``A'' is imported into ``B'', replaces some graphical elements of ``B'', thus enables the representation of some data attributes that ``B'' alone fails to convey. Suppose there are 5 types of area in a park. A bar chart, Fig.4(h), illustrates their average price per unit area, a chord diagram, Fig.4(g), illustrates how passengers travel through each area. In Fig.4(i), the bar chart take the place of node segments in a chord diagram, resulting in a novel and informative visualization      . Some actual examples are the heat map mapped upon the steams in a theme river\cite{wu_opinionflow:_2014}  and usage of glyphs to enhance the meaning of nodes in a multidimensional scaling plot.\cite{chen_peakvizor:_2016}

\textbf{Dependence} is an one-way relationship. If one visual unit ``A'' is dependent on ``B'', it means that ``A'' reveal some information that results from the visualization of ``B''. For example, a multiple dimensional scaling (MSD) map, Fig.4(j), shows the similarity between each restaurants in a city. A heat map, Fig.4(h), is then added to the MSD map to show the most common type of restaurants, which information can hardly be obtained from the dataset but quite evident from the MSD map, as in Fig.4(i). The biggest difference between "\textbf{enhancement}" and "\textbf{dependence}" is that \textbf{enhancement} still illustrate the data attributes in the dataset, while \textbf{dependency} reveals the new knowledge we obtain from adopting a previous visualization to the dataset. 

\subsubsection{Correlations between visual primitives}
The inner relationship between visual primitives is relatively simple.
A visual unit usually has 1 or 2 visual primitives. 

The relationship between the 2 visual primitives, if there are two, depends on the data attribute they indicate. 
\notsure{We identify three types of relationships between visual units: none, spacial, temporal, and logical.  The encoding of one primitive always has a high dependency on the encoding of another primitive. For example, in a chord diagram, the encoding of the arcs should be explained before the line connecting them. }


\begin{figure}[tb]
 \centering % avoid the use of \begin{center}...\end{center} and use \centering instead (more compact)
 \includegraphics[width=\columnwidth]{unit_relationship}
 \caption{Illustration of the 4 types of relationship between visual units}
 \label{fig:relationship}
\end{figure}

\subsubsection{Correlations between visual channels}
The relationship between channels might be the most complicated relationship in our model. Since different channels are encoded with different information, they are usually separated and have no logic dependency upon others. However, an arbitrary explaining order of visual channels may lead to an unefficient information delivery, especially when a mark has multiple channels, which is common in an advanced visualization. 

Therefore, we define two metrics to order the explaining of visual channels: \textbf{the complexity of their encoded information} and \textbf{saliency of their visual appearance}.

First, a proper explanation should follow the order of decreasing visual saliency.\cite{cleveland_graphical_1984} Even though different channels have intrinsically different perceptual salience and channel with high salience will suppress the expression of other, such salience strength can be influenced in a task-dependent manner. \cite{nothdurft_salience_2000} By introducing the channel with high saliency first, we remove it from the task list in our mind, decrease its saliency and give other channels more chance to attract limited human attention. (maybe introduce some concepts such as salience map, pre-attentive stage, and focused attention stage) \par
 Second, we should follow the order of increasing complexity. “Easy to difficult” practice has been long used and confirmed to be effective for learning new tasks\cite{bliss_effects_1992}.\par
 Based on our survey, there are five common visual channels: color hue, size, position, shape, color saturation. Sorted in the increasing complexity order, it is color hue-color saturation-position-size-shape, while sorted in the decreasing visual saliency order, it is position-color hue-size-shape-color saturation \cite{munzner_visualization_2014,cleveland_graphical_1984}.  \par
In our system, we choose position-color hue-color saturation-size-shape as a trade-off between these two metrics. But we do allow and recommend the users to define their preferable sequence based on their situation. 


\subsection{Analysis of existing visual distraction}
We aims to introduce a new visualization design in a visual method, more specifically, in the form of a slide show. To better inform the crafting an attractive and effective explanation, analyzing the existing visual distraction is necessary. 
\textbf{add a fig}
From our observation, we identify two kinds of visual distractions: visual distraction from context and visual distraction from sibling channels (sibling channels refer to the channels belonging to the same mark). \par

\subsubsection{Visual distraction from the context}
This kind of distraction has been widely discussed in the field of object detection and human visual attention. \cite{nothdurft_salience_2000, standage_modelling_2005}Its intensity is mainly  determined by spatial distance and appearance similarity. \cite{wolfe_guided_1994}For example, when we try to focus on a green rectangel, a red triangel near by it can lead to visual distraction. And the intensity of such distraction is determined by the distance and the appearance similarity between the two graphics. Focus + Context, which might be the most popular techniques for this problem, make uneven use of graphic resources to discriminate focus from their context. At the same time, adding dynamic changes to focus elements has also been demonstrated as effective under various conditions\cite{waldner_attractive_2014}. 

\subsubsection{Visual distraction from sibling channels}: A visual primitive usually has more than one visual channels. Thus, when recognizing one primitive, the channels with high visual saliency can significantly influence the expression of other channels. For example, color can be a strong noise when focus is supposed to be the shape.


\subsection{Design considerations of narrative sequence}


\subsubsection{Channels}
As discussed in section 3.1.4, when explaining channels, we should take information complexity as well as visual saliency into account. 

As for one channel, the narrative explanation depends on the type of the channel. As defined by previous work, therer two foundamentally different kinds of channels. The identity channels tell about what-where, while the magnitude channel tell how much. For a magnitude channel, one or two extreme examples will be enough for explaining, while for an ordered channel, introducing each category one by one might be a better option.

\subsubsection{Units} 
As discussed in section 3.1.2, there are four types of relationships between visual units, and they will influence the order of a narrative explanation. Thus, we display the correlations between units in a tree diagram where a child node is the enhancement/dependence of its parent node and sibling nodes have logic dependences. When explaining these visual units, we can simply follow a deep first search (DFS) order to visit all the visual units.

\subsubsection{Non-linear sequence}
so far, all the narrative explanation we discussed is linear. We move from one channel to another channel, then from one primitive to another primitive, then from one unit to another unit. However, reading a lengthy, extremely detailed instruction maybe tedious. A good narrative explanation should include non-linear design, allowing users to skip uninterested parts, go back to previous information and freely switch between different parts. Also, users should be allowed the flexibility to choose explanations at different levels of details. 


\section{Design Considerations}
In this section, we first describe our understanding of two groups of end users, i.e., editors and general audience. Then, we distill design tasks to guide our design and development of Narvis.

\subsection{User Perspectives and Methods}

Narvis aims to offer an efficient, expressive and friendly authoring tool for experts in data visualization, assisting them to create a slideshow to introduce advanced visual design to general audience.  
Hence, we identify two different user perspectives: the editors and the general audience perspectives. Editors are visualization experts who have the need to create a slideshow to present visual designs. General audience have no prerequisite for visualization. They gain understanding of a visual design through the slideshow created by the editor. 

To understand the current practice of making slideshows and the experience of reading tutorials, we collaborated with two teaching assistants (TAs) of a Data Visualization course and seven undergraduate students (UGs) taking this course. The two TAs are postgraduate students whose research interests are information visualization. Their duty of this course involves making slides to introduce visual designs from major publications in the field of visualization. The slides should cover fine-grained description to help students review them after class. The seven UGs have no prior experience in visualization, and have taken this course for no more than one month.  

We began by conducting semi-structured interviews with TAs, whom we identified as editors, and UGs, as general audience. During the interviews with TAs, we asked their workflows of making slideshows and explaining visual design. To identify opportunities for Narvis, we also asked them to enumerate a list of challenges faced in the workflows. The interviews with UGs are semi-structured as well. We asked their comments in reading the slideshows and attending course lectures. Then, we used mind-mapping to find clusters in their comments that defined goals for an ideal slideshow. 


\subsection{Design Tasks}
Based on our observations and the interviews, we categorize six design tasks to guide the design of Narvis. Three tasks, denoted as DE, are originated from the interview with editors, i.e., the two TAs, and other three tasks (DA) are from general audience (UGs).

%Our premise is that the editor is familiar with a visualization design but unclear how to educate his audiences in an efficient way. Since the background of the audiences cannot be guaranteed, we assume they all have no prior knowledge in data visualization. 
\noindent
\textbf{DE1. Emphasis on  efficiency.}
TAs used presentation tools, such as Power Point\footnote{https://office.live.com/start/PowerPoint.aspx}and KeyNote\footnote{http://www.apple.com/keynote/} to introduce visual designs. However, these tools are for general purpose and not tailored for visualization presentation. 
For example, "focus + context" techniques are widely used in data visualization to guide the user’s attention to the region of interest \cite{doleisch2003interactive, kosara2002focus+}. It requires exaggeration or suppression of the visual channels, like hue, luminance , sharpness, or size, of graphical elements, which are hard to perform in these common presentation tool. 
% Moreover, with the large number of visual encodings existing, she fails to determine an optimized order for explaining them. Even though she has many years of experience in data visualization, she never think about this issue before. 
%Splitting a visualization into fine-grained graphical elements and combining them into logic flow are tedious and time-consuming with existing tools. 
%Therefore, automatic decomposition of visualization and composition of graphical elements are necessary for facilitating design and production of a slideshow.
% TAs require an efficient approach to generate a slideshow introducing visual design.

\noindent
\textbf{DE2. Suggest options.}
People with extensive experience in designing data visualization might have little knowledge about how to present a visual design. 
Thus, suggesting design options to them can be helpful. Many presentation tools already offer this kind of service. For example, Power Point Designer\footnote{https://support.office.com/en-us/article/About-PowerPoint-Designer-53c77d7b-dc40-45c2-b684-81415eac0617} automatically generate a list of professionally designed layouts based on the contents. However, these suggestions focus on aesthetic issues, and give no special consideration for presenting a data visualization.
For example, how to compose a clear narrative sequence from all the visual grammar employed.
\siwei{your logic is: other tools have no special consideration for doing sth. Reviewers may think: why they should have such consideration? You should emphasize these consideration is important in visualization because of some reason.}
%The complicated relationship, spatial and logical alike,  between different graphic elements is one of the biggest barriers that hinders the effective presentation of a visual design. Howe like Power Point Designer"
%By encouraging users to refine the clusters of graphic elements, to figure out relationship between visual units by editing a unit tree, and to modify the narrative templates we offered, we aim at motivating users to sort out the structure of a visualization, which will then result in a better-organized narrative sequence. 

\noindent
\textbf{DE3. Collect feedback.} \textit{``When students read my slides, I do not know whether they can follow the logic, or whether the slides cover enough details for them to grasp the visual design.''}, one TA commented. 
% When giving a lecture, TAs are able to understand how their slides and logic are accepted through in-class interactions. 
Collecting feedbacks of audience is crucial for editors to revise their slideshow, making it more understandable and attractive.  

%We implement a click collector which will automatically record click activity when audience view the explanation in our system. The click stream data can reveal information about viewing order, for example, the audience might skip one slide or go back to revisit another, the time spent on each slide, the degree of understanding at each part.The audience can also write down their comments while reviewing such an introduction slide show. Editor can adjust the narrative sequence as well the level of details based on these feedbacks. 
\noindent
\textbf{DA4. Avoid information overload.} 
All UGs complained that they had experienced information overload in reading slides. 
%Though they might separate an informative slideshow into parts and read one part at a time, they might miss some when the information in one slide is overwhelming.
% We distinguish the amount of information in the slideshow and in one slide. Audience  However, 
When the information in one slide is overwhelming, it is common for they to miss important visual encodings. 
The slideshow should be well designed to ensure that the amount of new information in each slide is appropriate. 

\noindent
\textbf{DA5. Avoid unconscious ignorance.}
Experts in data visualization, i.e. the TAs of the course, prone to treat some visual grammars as self-evident that need no  explanation. However, the lack of information confuses the UGs, who have no prior knowledge of visualization. 
Considering the importance of information integrity to a comprehensive slideshow, we need a mechanism to guarantee that all visual grammars are explained. 

\noindent
\textbf{DA6. Keep the sense of overview}
Grouping slides into sections, and inserting visual notice, like a progress bar, to indicate the overall structure is an effective and widely used presentation skill \siwei{if you say they are widely used, please cite}. However, partly due to the lack of a clear narrative structure, this technique is rarely used in the presentation of a visualization, in spite of the demand from audiences. \siwei{do not say the technique is rarely used in visualization because it is not convincing. say how the technique can help visualization presentation, therefore it is important.}




\section{Narvis: System Design and Implementation}

% Thus, it has two kinds of users: editors, the data visualization experts who use Narvis to create an explanation slideshow, and audiences, the general audience who watch the slideshow created by ediors.

With the considerations above in mind, we implement the workflow of Narvis consisting of three phases (Figure \siwei{ref}), i.e., Automatic Analysis Phase, Manual Editing Phase, and Viewing Phase.

% The workflow of Narvis includes three phases (Figure \siwei{ref}):  In Automatic Analysis Phase, the system accepts the input visualization, extracts graphical elements and classifies them into groups for further edit. 
% In Human Editing Phase, editors will be involved to modify the output from the Automatic Analysis Phase, and use the templates Narvis provide to craft a explanation slideshow. 
% In Reviewing Phase, audience can assess the slide show generated in Human Editing Phase. Their click activity and comments will be recorded and send to the editors, helping them improve the quality of the slide show. 

\begin{figure}
 \centering % avoid the use of \begin{center}...\end{center} and use \centering instead (more compact)
 \includegraphics[width=\columnwidth]{overview}
 \caption{The system overview}
 \label{fig:overview}
\end{figure}


% two editors and seven general audience, respectively. Both interviews are semi-structured. 

% We should take editors and audiences alike into consideration for the design of Narvis. We conducted a interview with 3 experts in data visualization, asked what they could expect for an authoring tool for crafting an introduction slideshow. 
% We also constructed an interview with 7 undergraducate students. They have no prior experience in data visualization, and they just come back from a lecture about data visualization, which, as they said, ``totally confused'' them. They talked freely about what has confused them.
%It is interesting that there are some overlap between the requirements from editors and that from audience. For example, they both require slide  


% Based on this interview, as well as  our own experience and examination of literature, we settle on six design considerations(DCs) for the aspect of editors (DE) and that of audiences (DA). 

 %\textbf{DA7. Information repetion}
 %When introducing a visulization design, it is common for the audiences to forget previous information or lose the sense of overview. Information repetition, also called as redundancy, helps with visualization recall and understanding. \cite{borkin_what_2013}}
 %by providing simplified sentence pattern for annotation adding, we skip such parts in purpose.

%The audience of the generated narrative explanations are ordinal people, and they usually have no interests in understanding the algorithm employed in a visualization design. For example, the comprehension that a transition line indicates public attention shift from one topic to another is enough for them. Explicating how to quantify public attention shift only bores the audience, even scares them away.

\subsection{Phase1: Auto Analysis}

% The auto analysis has two parts: one for input image and one for input text. It automatically extract the graphic elements and divide them into different cluster, facilitating later editing.(DE1) Note that the textual input is not necessary but it provides hints when editors add annotations manually in the Human Editing Phase.(DE1)

The input of Narvis includes two parts: one image presenting a visual design (mandatory) and a piece of text describing the design (optional). In this phase, Narvis accepts the input image and extracts graphical elements to facilitate further authoring. If text is provided, editors are able to add and edit annotations in the Manual Editing Phase.

\subsubsection{Analysis of input image}
% The auto analysis of input image has three main steps. It first detects all primitives that it finds in the given image and also detects any labels that are present in the visualization. It will then cluster objects that are spacially linked and extract non-target objects. Finally, it will fill in any empty spaces left inside objects from extraction with the appropriate color so as to show the target object in its entirety.
The analysis of input image includes three steps, i.e., object detection, object clustering and object recovery.

% The first step, object detection, is done by iterating through all the pixels on the bitmap. 
\textbf{Object detection.} Narvis iterates through all pixels in the input image. At every iteration, we first check to see if this pixel has already been tagged as part of an object. If not, we know that this pixel forms \siwei{a part of a new object.} We explore the colors of the neighboring pixels, where the neighbors are chosen such that the distance between the current pixel and a potential neighbor \siwei{is less than 3}. If the difference in color between a neighbor pixel and the current pixel is less than a threshhold, the neighboring pixel is tagged as part of the same object. Once all neighbors have been classified as either part of the same object or not, we choose another pixel that was classified as part of the same object and apply this algorithm again. This is a modified BFS algorithm and allows us to identify all unique ojects in the given visualization.

\textbf{Object clustering.} Once all the objects have been detected, we have to extract a target object. To extract an object means to only select the pixels that are classified as part of this object, so we should remove all objects that are not part of this object and we should extract objects that are inside our target object.It is trivial to set all pixels that are not within or part of our object to have color white. For objects that are inside our target object, i.e. those objects that are clustered with out target object, we will first detect that object then programmatically change its pixels to white.

\textbf{Object recovery.} Once we have completed extraction, we have the issue of these white spaces. The reason this is an issue is because an extracted object might have been dividing two objects, and so when it is extracted, we lose the boundary between our target object and another object, which can cause confusion as to whether that white space should be colored in or not. To solve this boundary problem, we create a queue of the white spaces, with each data point giving the starting and ending point of that space. We then look at the intervals between enclosed white spaced objects, if that interval is above a threshhold, we take that white space to not be part of our object. If it is below our threshhold, then we enclose the white space with the target objects color, creating a boundary for it. The main difference is that for objects not within our target object, we do not create a boundary, whereas objects within our target object are enclosed with the target objects color.

\subsubsection{Analysis of input textual description}
For the input textual description, we offer a basic text detection and classification algorithm, which uses a dictionary of terms that are highly correlated with certain channels. E.g. the word "length" is highly correlated with the size channel. To do the text detection, we first classify each sentence depending on whether it contains any of the key words in our dictionary. If it contains a key word for one of the channels, the sentence is tagged as being a description of that channels visualization. Once we have tagged all the sentences, whenever a channel is selected, we show the entire text that was inputted and highlight the text that has been tagged as descriptive of that channels visualization.

The algorithm we proposed is a compromise between efficiency and performance. At this time point, it is limited for image with high quality and clear edges, but its performance can be improved by applying a more advanced algorithm, such as the method based on patch detection and clustering mentioned in Revison\cite{savva_revision:_2011}.

\subsection{Phase2: Manual Editing}
% In Narvis, designers specify visualizations as a hierarchy of visual units with visual properties
The workflow of this phase includes three steps, i.e., visual unit generation, visual unit organization and template integration, as illustrated in Figure \siwei{ref}. We further introduce a library of templates for \siwei{sth}
% We introduce three panels in this phase acting as three steps in the workflow to allow editors  

\subsubsection{Visual Unit Generation} 
After graphical elements are extracted and clustered based on visual representation, each cluster appears as a tabbed view in the Source Panel (Figure \siwei{ref}). By default, each cluster includes all graphics elements belonging to one visual unit. However, editors might want to redefine visual units by semantics. For example, \siwei{add an example with figure}. 
Narvis allows editors to \siwei{drag and drop, and other interactions} to ease the process of visual units generation. 

% editors are able to manually edit and organize these elements to author a slideshow.  

% \subsubsection{The interface for editing}
% The interface of editing mode is composed of the following panels. We arrange the position of these panels and their content to better match the observed authoring work flow: refine clusters, bind with templates, organize the structure of visual units, modify templates, add annotation. 


% \textbf{\textit{Source Panel}: extracting and organizing graphic elements}
% FigSource is a tabbed panel where the extracted graphical elements are associated with different tabs based on the pre-cluster results. The user add, delete, modify the graphical elements associated with each tab, making sure that 1) All the graphical elements of the same visual unit is with one tab 2) every graphical element belongs to one and only one tab. For each tab, which actually equals to a visual unit now, the user call a template from our library in a drop-down list. 

\subsubsection{Visual Unit Organization} 
\textbf{\textit{Tree Panel}: clarify the structure of a visualization}
\textit{Unit tree} panel tends to motivate the users to figure out the relationships between different visual units througn interacting.  
In the \textit{Unit Tree} panel, all visual units are shown as tree nodes.
With Interactions as simple as dragging and dropping, the users organized and display the structure of the visual units in the panel, like what we have discussed in section 3.3.(DA5) To help people better identify the relationship between visual units, which might be a new concept for them, we include a tutorial here. Even though learning the relationship between visual units requires extra effort and time, we believe it is worthwhile since it can give people a better understanding about the structure of a visualization. Based on the tree diagram, Narvis will refresh the narrative sequence of visual units. 

\subsubsection{Template Integration} 
\textbf{\textit{ Unit Panel \& Editor panel}: personalized modification}
Narvis provides templates to achieve high efficiency, but it also allow the users high flexibity to modify these temples, thus guaranteeing the expressiveness of this system. 
 
Editors can edit a template in \textit{Unit Panle} by selecting a node on the \textit{Tree Panel}.For each visual units, the template enumerates possible encodings and leave the users to delete unemployed one, thus eliminating the unconscious missing of crucial information. (DA4)It also recommends a narrative sequence based on the metrics we mentioned in section 3.1.4. (DE1, DA3, DA5) 
 
In the editor panel, users get further to access the \textit{grammar} of each visual primitives, add a short annotation to describe it(DA6), refine or remove the animation we embedded in a template. 

\subsubsection{A library of templates}
We propose a library of templates for the narrative explanation of a visualization. A templates is a set of slides that tends to introduce a visual unit, which can be descibed as an orthogonal combination of a visual primitive and a construction rule, as shown in tab.1. Since advanced visualization design is the assembly of miscellaneous visual units, we conjecture such templates can achieve a high level of efficiency for the explanation of a viusalization. (DA1)Meanwhile, allowing users a high flexible, friendly interface to edit offered templates, Narvis maintains a considerable level of expressiveness and accessibility. 

\textbf{Types of templates}

The initial set of templates provided by Narvis can be described as  a 9*4 matrix, 9 types of visual primitives and 4 types of construction rules. Narvis is extensible, new templates can be added by its developer through programming, or by end users through uploading their modified templates. At the same time, all the supported templates are classified into a certain cell of the 9*4 matrix, so as to avoid overwhelming users with a cornucopia of confusing options.

\textbf{Templates design}

We apply the analysis and theory model in section 3 for the design of templates. A template has three components: 1) a well-considered narrative sequence for visual grammar explanation, which  is discussed in section 3.3 and reveal encoding grammar gradually(DA3); 2) Embedded a series of narrative techniques such as attention cues, animated transitions, information repetition, to orientate visual attention and facilitate perception; 3) Formatted sentence for annotations (DA6) that will be gradually disclosed in the slide show. (DA3)

With a visual unit, more specifically, a set of graphic elements, as input, a templates will generate a series of slide show and each slide is responsible for the explaining of one visual grammar. A visual properity show on a slide only after its grammar has been explained. For example, if we havn't explain the encoding of color, all the object in current slides will be gray. These slides are sorted based on the narrative sequence we discussed in section 3.3. The graphical elements in different slides, which might have different visual appearance, are perceptively connected through morphing animation.

\textbf{Animation embedded in templates }

Narvis provides 8 types of animation, implement them in templates based on their effects on human attention and perception(DA1), which has been widely discussed in previous work.\cite{robertson_effectiveness_2008, waldner_attractive_2014, heer_animated_2007}We also provide an novel decomposition animation at the beginning of the introduction slide show to engage the audience as well as to help them get a sense of overview.

\begin{table}[tb]
  \caption{A summary of animation provided}
  \label{tab:animation}
  \small
  \centering
  \begin{tabular}{p{1cm}|p{0.9cm}|p{0.9cm}|p{0.9cm}|p{1.5cm}|p{0.9cm}}
  \toprule
 \textbf{Animation} &\textbf{Engaging} & \textbf{orientate attention} & \textbf{perception} &\textbf{working scenario} &\textbf{ref} \\ 
  \midrule
  \textbf{Morphing} &\checkmark & \checkmark &\checkmark & grammar of size, grammar of shape & \cite{ruchikachorn_learning_2015} \\ 
  \midrule
  \textbf{Blur} &   &\checkmark  &   & focus+context & \\ 
  \midrule
  \textbf{Flicker} & & \checkmark &  & focus & \\
  \midrule
  \textbf{Motion} & \checkmark & \checkmark & \checkmark &grammar of position &  \\
  \midrule
  \textbf{Zoom-in/out} & \checkmark &\checkmark &  & focus&  \\
  \midrule
  \textbf{Annotation} &  & \checkmark &\checkmark &   textual explain & \\
  \midrule
  \textbf{Fade in/out} &  & \checkmark &  & & \\
  \midrule
  \textbf{Decompose} & \checkmark &  &\checkmark & Show how a visualization is composed by visual units & A novel design bu us \\
  \bottomrule

  \end{tabular}
  \vspace{1mm}
\end{table}


Animation is a double-edge sword, which introduces both benefits and pitfalls. We are not discussing the effects of animation here. Editors can choose to remove these animation if they prefer an abstract slide show or they are suspicious of the effects of animation. 


\subsection{Phase3: Viewing}
\subsubsection{The interface for audience}
The interface of audience is composed of two panels.

\textbf{\textit{Gallery:}the collection of generated slide show}
 
Gallery exhibit all the slide show produced by editors and saved in Narvis. Every slide show is presented by a image, the visualization it tends to explain. By clicking on the image, users can watch this slide show in the \textit{Screen} panel. 

\textbf{\textit{Screen:}  review and comment}
Every slide show displayed in \textit{Gallery}is a series of slides, each of which is responsible for the delivery of one simple encoding information, for example, the horizontal position indicates time. In the \textit{Screen} panel, users click buttons to move forward or backward to view these slides. 
\subsubsection{Generated Report}
The report visualize the click activity of audience in the form of a stacked bar chart. The heigh of the bar indicates the time spent on watching this slide. If audiences go back to revist a slide while viewing, a bar will be stacked on the top of previous one. If there are animation in the slide show, a white line will be drawn on the bar chart, indicating the animation playing time of each slide, thus can indicate whether an animation is too fast or too slow. (DE2)
\subsection{A working scenario}
Jessica has extensive experience in the field of data visualization, and has implement a visual analytics tool in a review service website based on the design of OpinionSeer\cite{wu_opinionseer:_2010}. To help audience better understand this design, she needs to publish a tutorial accompanied with it.
% Here, we demonstrate the utility of Narvis through the creation of a narrative explanation of a visualization design that published in TVCG (http://ieeexplore.ieee.org/stamp/stamp.jsp?arnumber=5613449). 

% \subsubsection{Motivation}Jessica, a expert in data visualization, realizes the promising prospect of implementing this visualization design in an on-line review service website like Yelp. (pg student to classmates? prof to a journalist? prof with collaborators from other fields?  )\par
% However, she has to present this design in a general meeting to get the financial support. She is familiar with this visualization design, yet not sure whether her audiences, with little knowledge about data visualization, can quickly learn this visualization design from her introduction. Without clear awareness of the visual encodings, audience can hardly identify the interesting patterns this visualization reveals and realize its utility.  
% A traditional way is to add highlights and annotations in the original visualization. But Jessica worries about the effectiveness of information delivery. 
% She considers about applying "focus + context" techniques,  but it will introduce some complicated operations for her, who has no experience in image editing. 
% Moreover, with the large number of visual encodings existing, she fails to determine an optimized order for explaining them. Even though she has many years of experience in data visualization, she never think about this issue before. 
% She then turns to Narvis for help. 

% She first imports visualization design, in the form of a PNG file, and the corresponding text description, which she directly copy from the paper, into the system. 
First, she loads the screen-shot of her system, as well as a piece of textual description, into Narvis.
% After a few seconds, the system automatically detects graphics elements with a edge detection algorithm and cluster them based on their features. The algorithm cannot be perfect. For example, it puts geo ring and calendar ring in the same cluster based on their similar appearance. 
After a few seconds, the system automatically extracts the graphics elements and clusters them based on features. As Figure\siwei{ref} shows, Jessica obtains four clusters. 

Then, she defines visual units based on clusters. By default, each cluster includes all graphics elements belonging to one visual unit. However, she observes that geographic ring and calendar ring are in the same cluster due to their similar appearance. Therefore, she divides it into two clusters, containing geographic ring and calendar ring respectively.
% then refines the clustering results, making sure every graphic elements in one visual unit is put in one cluster \textbf{figure xxx}. Finally she obtains four clusters.

Next, she chooses narrative templates for each visual unit. 
%Some visual units, such as the triangle scatter plot, are novel and no narrative template in the library is able to match. Therefore, Jessica chooses the template of regular scatter plot for best match.
% for triangle scatter plot, which differs from the triangle plot in the encoding of position. 
Moreover, Jessica edits the narrative templates based on her design. 
% In the templates, we enumerate all the possible visual encodings. 
She goes through all four templates in the ``\siwei{what is in-unit}in-unit'', and deletes the visual channels with no encodings, such as \siwei{sth}. 
Through drag and drop, Jessica further organizes the structure of the unit tree based on the relationships between units. For example, \siwei{some example}

Jessica further improves the quality of animation by adding annotations and strengthening the binding between data and graphic elements. 
% When adding annotation to a certain channel, the related text will highlight in the text area, aiming to offer a better user experience.   


To refine the readability of the tutorial, Jessica asks several friends, who have no experience in data visualization, to watch the tutorial before release. Narvis collects their viewing behavior from click activities, generates statistics results, and visualize it in the form of stacked bar chart , which helps Jessica answer questions like \textit{``which slides do they skip?''}, \textit{``which slides do they review several times?''}, and \textit{``which slides do they stay for a long time?''}.
% These click sequence data provides cues for Jessica to strengthen and refine the readability of the tutorial. 

% revealing information such as, ,   




\section{Evaluation}
We conducted a user study to evaluate Narvis, and gained insights on how the authoring experience and output would compare to slideshows created with general presentation tools. Our study was a between-subjects design with two sample groups: one group of participants used Narvis, the other group used Powerpoint, which is widely used to create presentation slideshow. 
We report our qualitative observations during the authoring process, and provide insights on the quality of the slideshows generated from both groups.
\subsection{Participants}
We invited 4 experts in data visualization to this user study as editor participants, denoted as PC1 and PC2 for control group, PE1 and PE2 for experiment group. All of them have more-than-one-year experience in the design and implementation of data visualization. We also sent emails to students in the data visualization course we mentioned before and recruited 20 volunteers to evaluate the quality of the generated slideshow as audience participants, denoted as PAs. 
%\textbf{Audience }are novice in data visualization. They will review the slideshow produced by the experts, rate it, give subjective comments, and answer a series of questions to check their understanding of this visualization.\par
%For editors, we have 4 postgraduate students, aging between 22-30, and all of them have more than one year experience in data visualization.\par
%For audiences, we have 20 under graduate students, whose majors vary from business to biology. 
%According to the questionnaire, none of them have accessed advanced data visualization before. Only 13\% students know the tree map, and none can give a accurate explanation of theme river with topic splitting and merging.  \par
\subsection{Material}
We extracted the visual design and the corresponding literature description from  a visualization design paper ``TextFlow: Towards Better Understanding of Evolving Topics in Text''\cite{cui_textflow:_2011}.
We chose \textit{TextFlow} based following considerations. First, it's not too difficult for a novice but still a novel design that requires extra effect to clarify.
Second, it is a typical abstract data visualization that is fully consist of graphical element, which is in the coverage of our edge detection algorithm. 
Third, it visualize evolving topics in social media, which is an interesting topic and can increase the engagement of audiences. 

This visualization design conveys multiple level results of topic evolution analysis: a set of topics
with splitting/merging relationships among each other, which encodes a series of topic flows, a set of critical events, which encodes glyphs, and the keyword correlations, which encode threads.  

\subsection{Procedure}
\subsubsection{Generating Slideshows}
We ran 90-min long sessions for the four participants separately. This session consist of 3 phases: (1)\textit{Learning Phase}, (2)\textit{Sketch Phase}, (3)\textit{Authoring Phase}.

In the \textit{Learning Phase}, participants read the literature description we extracted from the paper, which offered a detailed description of the visual design with diagrams. This phase ended when the participants reported the experimenters that they finished reading and understund this visual design. 
This phase took 15 min, 14 min, 17 min and 13 min for 4 participants respectively.

In the \textit{Sketching Phase}, participants were asked to sketch ideas for introducing \textit{TextFlow}. They were encouraged to give considerations to (i) convey the insight to the people with less experience in data visualization; (ii) organize a clear narrative structure; (iii) think about additional annotation and animation required. Participants are asked to think aloud and experimenters are present in the room to observe. 

In the \textit{Authoring Phase}, participants implemented the ideas in their sketch as detailed as possible in an one-hour-long session to produce a narrative slideshow that can be self-explainatory. We send each participant a PNG file and a txt file as the raw material for authoring. Participants in control group use Power Point, a presentation making tool that all the participants are familiar with. In experimental group, before authoring, experimenters demonstrate the working flow of Narvis through an automatic step by step tutorial included in Narvis. This training lasted about 15 min and is not counted in the one-hour authoring session. Participants are also allowed to ask additional questions in the authoring phase.

\subsubsection{Evaluation Methods}
The evaluation focus on two parts, the authoring experience of Narvis and the quality of the generated slideshows. We report our observation of the authoring experience based on a interview with the two PEs and the video we took during the authoring session. 

To evaluate the slideshow generated from both groups, we first analyzed the slideshows and reported some quantitative observations, such as the number of slides.
Then, to get an independent opinion, we asked 20 PAs to evaluate the generated slideshows. To eliminate the error introduced by other variables such as the different watching environment, this part was conducted in a website we built. Each PA was randomly assigned a slideshow when visited this website. They watch all the slides by clicking two buttons, ``next'' and ``previous'', and their click activity was automatically recorded by a background program.  After watching the slideshow, they were asked to finish an online questionnaire composed of 2 parts: 1) a quiz about the visual design of ``TextFlow''(with a full mark of 5);2) rate the slideshow  1 (very poor) to 5(excellent) at various aspects. 

\section{Results}
%We analyzed the following material: 1) video and notes that the experimenters took during the user study session, which the participants consented to. 2) the slides and the sketch created by participants, 3) the interview with the editor participants, 4) the ranking, comments, answers, click stream data from the audience participants. 

\subsection{Generated Slideshow}
We obtained 2 slideshows from the control group, denoted as SC1 and SC2, and 2 slideshows from the experiment group, denoted as SE1 and SE2(see in Figure~\ref{fig:user_study}). 


\begin{figure*}
 \centering % avoid the use of \begin{center}...\end{center} and use \centering instead (more compact)
 \includegraphics[width=\linewidth]{user_study}
 \caption{The slideshows produced by (a)Narvis and (b),(c)Power Point to introduce a visual unit, thread, in \textit{TextFlow}\cite{cui_textflow:_2011}. Note that (b) and (c) both miss the visual grammar of thread color and (c) forgets to mention the visual grammar of wave bundling length. }
 \label{fig:user_study}
\end{figure*}

\noindent
\textbf{Observation from experimenters}

Here, we report our observation of the 4 slideshows baed on 1)the generated slideshows and their sketch; 2)the video and notes we took during experiments; 3) the click activities of the PAs; 3) interviews with participants.

SE1 and SE2 are similar since they were conducted with the same templates in Narvis. However, SE1 included all the animations we embedded in offered templates while SE2, whose creator preferred an abstract introduction, deleted most animation. SC1 explained the visual design with long, detailed textual description that was formatted with bullet points. SC2 mainly used symbol-based annotation for explaining, and re-editted the image we offered in Power Point . Table~\ref{tab:slides2} gives a quantitative report of the four slideshows.

Information omission occurred at all four sketches, even though their creators were given the freedom to check with the provided material. For example, three sketches (the sketches for SC1, SC2, SE1) failed to mention the visual grammar of the size of glyph, two sketches (SE1, SE2 )omitted the visual grammar of the color of thread. 3 mistakes out of 4 got corrected in SE1 and SE2. When editing in the \textit{Unit Panel} to delete unemployed channels, the two editors both felt unsure about whether certain channels should be deleted, which helped them correct their omission. 
For SC1 and SC2, only one mistake got noticed and corrected by its editor while the other two remained the same.

With the same authoring time, SE1 had 29 slides, SE2 had 22 slides, SC1 had 7 slides, SC2 had 3 slides. PC2(the creator of SC2) spent most of the time to add symbol-based annotation, re-edit the image to realize techniques such as zoom-in, thus had little time to organize the textual annotation. Huge blocks of text were put arbitrarily in SC2.

The total time required to read SEs was not significantly longer than that for SCs. This was out of the experimenters' expectation, especially when considering the number of slides included. 
In SES, the average staying time at each slide was evidently lower than that in SCs, which might come from the short length of text. 

\noindent
\textbf{Evaluation by PAs}

Table~\ref{tab:slides2} presents the results of the questionnaire. 

\begin{figure}
 \centering % avoid the use of \begin{center}...\end{center} and use \centering instead (more compact)
 \includegraphics[width=\linewidth]{clickstream}
 \caption{The visualization of clickstream data when PAs watch the slideshows created by PE1(a), PE2(b), PC1(c), and PC2(d), respectively. In the line charts (top), each line is the watching trace of one PA. x-axis indicates the number of slides and y-axis indicates the total time used from beginning. In the stacked bar chart(bottom), a series of bars represent the average time PAs spent on watching each slides. }
 \label{fig:clickstream}
\end{figure}



\begin{table}
  \caption{A summary of 4 slideshows}
  \label{tab:slides1}
  \small
  \centering
  \begin{tabu}{p{2cm}|p{0.9cm}|p{0.9cm}|p{0.9cm}|p{0.9cm}}
  \toprule
 \textbf{} &\textbf{SE1} & \textbf{SE2} & \textbf{SC1}& \textbf{SC2} \\ 
   \midrule
  \textbf{Number of Slides } & 29  & 33 & 7 & 3 \\ 
 \midrule
  \textbf{Average Reading Time(Total,s)} & 327.05 & 156.78 & 169.33 & 128.84\\ 
 \midrule
  \textbf{Averrage Reading Time(Per Slide, s)} & 11.27 & 7.09 &24 & 42\\ 
   \midrule
  \textbf{Information Missing (in Slideshow/in Sketch) }& 1/4 & 0/3 & 2/3 & 2/2\\ 
     \midrule
  \textbf{Average Length of Text (per Slide) }& 10.7 & 12.3 & 32 & 47\\ 
    
  \bottomrule

  \end{tabu}
  \vspace{1mm}
\end{table}



\begin{table}[tb]
  \caption{The questionnaire result}
  \label{tab:slides2}
  \small
  \centering
  \begin{tabular}{p{1.5cm}|p{0.9cm}|p{0.9cm}|p{0.9cm}|p{0.9cm}}
  \toprule
 \textbf{} &\textbf{SE1} & \textbf{SE2} & \textbf{SC1}& \textbf{SC2} \\ 
   \midrule
  \textbf{Quiz Score } & 3.75  & 3.17 & 2.6 & 3.0 \\ 
 \midrule
 \midrule
  \textbf{Readability} & 3.8 & 3.5 & 3.2& 2.75\\ 
 \midrule
  \textbf{Utility} & 3.875 & 3.375 & 2.4 & 3.35\\ 
   \midrule
  \textbf{Aesthetics }  & 4.125 & 3.4 & 2.1& 2.75\\ 
  \midrule
  \textbf{Attractiveness} & 3.9 & 3.3 & 2.2 & 2.5\\ 
  
  \bottomrule

  \end{tabular}
  \vspace{1mm}
\end{table}

For SEs, the PAs were excited about the animation applied and one PA even asked about the source code. No complain was made about the relatively long watching time for SEs, which might due to ``the transition is smooth and the structure is well organized''(from one PA) and the short staying time at each slide. PEs also appreciated that textual descriptions were brief and were separated into different slides.
We got valuable suggestions for the improvement of Narvis, such as the inclusion of interaction and the implementation of a progress bar to demonstrate when an animation will end. 

For SCs, the huge blocks of text, which appears in both SCs, got most complained. 
One PA commented that`` it is hard to read and I have to admit I skipped some parts, thus still confused about this design.'' They enjoyed the symbol-based annotation and the way the creator re-edited the image in SC2. However, such operation is time-consuming in Power Point, resulted in a short, unfinished slideshow with unformatted text. 


\subsection{Authoring Experience of Narvis}
All 2 Participants were impressed by the overall Narvis design, mentioning that the workflow was intuitive and that the interactions and animations were smooth. 

\noindent
\textbf{Learnability}
We confirmed that all 2 PCs were able to craft a slideshow with Narvis after a short training period. 

\noindent
\textbf{General comments from PEs}

\noindent
\textbf{Observation from }

jj
\section{Limitation and Discussion}
We are not pretending that Narvis is exclusive for all types of visualization design, considering the initial set of templates Narvis provides. However, by allowing users a high flexibility to create and edit templates, we believe its coverage  will quickly broaden as more and more users contribute their own templates to our library. 

To broaden the form of the input. Embedded in a visualization analysis tool. 

For evaluation: small sample size, only compared to powerpoint,
however, as with all qualitative studies with small sample size, these results should be treated with caution and DataClips warrants further evaluation to confirm if our initial insights apply more gener- ally. 

\section{Conclusion and Future Work}
In this paper, we present Narvis, an authoring tool for crafting introduction slideshows for the purpose of introducing new visual design. 
Inspired by previous work and our observation, we propose a constructive model for introducing data visualization. This model is realized as a library of templates in Narvis, which recommends narrative sequences for the introduction of different visual units and supports easy creation of different attention cues.To better guide the development of Narvis, we also interviewed target users to extract their requirements. 

We evaluate Narvis through xxxx,

and get the conclusion xxxxx,

In future work, we envision two main research directions. First, we will deepen our understanding of what makes compelling and comprehensive introduction slideshows for presenting visual design. We plan to ground our work on the data from Narvis, more specifically, the templates, slideshows uploaded by editor users and the click stream data from audience users. Second, we will improve the performance of Narvis by 1) adding url as a possible input source; 2)offering better support for organizing the structure of visual units; 3)

%% if specified like this the section will be committed in review mode
\acknowledgments{
The authors wish to thank A, B, C. This work was supported in part by
a grant from XYZ.}


\acknowledgments{
The authors wish to thank A, B, C. This work was supported in part by
a grant from XYZ.}

%\bibliographystyle{abbrv}
\bibliographystyle{abbrv-doi}
%\bibliographystyle{abbrv-doi-narrow}
%\bibliographystyle{abbrv-doi-hyperref}
%\bibliographystyle{abbrv-doi-hyperref-narrow}

\bibliography{template}
\end{document}

